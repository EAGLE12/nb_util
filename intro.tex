\chapter{序論}\label{ux5e8fux8ad6}

    原子レベルの物理現象を数式だけから理解するのは非常に困難である.例えば,粒子の
動きを解析する分子動力学法では,ニュートンの運動方程式に従い粒子の運動を決定する.
それらは,数値の羅列で表され直感的な理解が得られない.学習において自分の頭の中で
イメージを作ることは重要であり,理解を助長することができると考えられる.そこで,
本研究ではグラフィック機能に特化したProcessing
を用いて分子動力学法での粒子の振る 舞いの視覚化を行い,Interactive
な操作が可能なプログラムを作成する.また,それらを 手軽に扱えるようにWeb
ブラウザ上で動作可能なJavaScript への変換を行う.JavaScript
への変換はProcessing
上で行う事ができるがそれには問題点があり,それらの洗い出し
と解決策を検討する.さらに,作成したプログラムをライブラリ化し,解説を行うことで
今後の継続的な発展に繋がると考えられる.

    